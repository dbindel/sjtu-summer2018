\documentclass[12pt, leqno]{article} %% use to set typesize
\usepackage{fancyhdr}
\usepackage[sort&compress]{natbib}
\usepackage[letterpaper=true,colorlinks=true,linkcolor=black]{hyperref}

\usepackage{amsfonts}
\usepackage{amsmath}
\usepackage{amssymb}
\usepackage{color}
\usepackage{tikz}
\usepackage{pgfplots}
\usepackage{listings}
\usepackage{courier}
%\usepackage[utf8]{inputenc}
%\usepackage[russian]{babel}

\lstset{
  numbers=left,
  basicstyle=\ttfamily\footnotesize,
  numberstyle=\tiny\color{gray},
  stepnumber=1,
  numbersep=10pt,
}

\newcommand{\iu}{\ensuremath{\mathrm{i}}}
\newcommand{\bbR}{\mathbb{R}}
\newcommand{\bbC}{\mathbb{C}}
\newcommand{\calV}{\mathcal{V}}
\newcommand{\calW}{\mathcal{W}}
\newcommand{\macheps}{\epsilon_{\mathrm{mach}}}
\newcommand{\matlab}{\textsc{Matlab}}

\newcommand{\ddiag}{\operatorname{diag}}
\newcommand{\fl}{\operatorname{fl}}
\newcommand{\nnz}{\operatorname{nnz}}
\newcommand{\tr}{\operatorname{tr}}
\renewcommand{\vec}{\operatorname{vec}}

\newcommand{\vertiii}[1]{{\left\vert\kern-0.25ex\left\vert\kern-0.25ex\left\vert #1
    \right\vert\kern-0.25ex\right\vert\kern-0.25ex\right\vert}}
\newcommand{\ip}[2]{\langle #1, #2 \rangle}
\newcommand{\ipx}[2]{\left\langle #1, #2 \right\rangle}
\newcommand{\order}[1]{O( #1 )}

\newcommand{\kron}{\otimes}


\newcommand{\hdr}[1]{
  \pagestyle{fancy}
  \lhead{Bindel, Summer 2018}
  \rhead{Numerics for Data Science)}
  \fancyfoot{}
  \begin{center}
    {\large{\bf #1}}
  \end{center}
  \lstset{language=matlab,columns=flexible}  
}


\begin{document}
\hdr{2018-06-20}{2018-06-27}

\paragraph*{1: SEP vs SVD}
Suppose $A$ is symmetric with the eigenvalue decomposition
$A = Q \Lambda Q^T$.  Describe how to compute the singular value
decomposition $A = U \Sigma V^T$ in $O(n^2)$ time.

\paragraph*{2: Dropping dimensions}
Suppose $A$ is a matrix of pairwise squared Euclidean between objects
in a $d$-dimensional space, i.e.~$a_{ij} = \|x_i-x_j\|_2^2$.  The
classic {\em multi-dimensional scaling} algorithm for reconstructing the
points (up to translation and rotation of the coordinate system) is:
\begin{itemize}
\item Apply double centering: $B = -\frac{1}{2} J A J$ where
  $J = I-ee^T/n$ and $e$ is the vector of all ones.
\item Compute the largest $d$ eigenvalues $\lambda_1, \ldots,
  \lambda_d$ and corresponding eigenvectors $Q = [q_1, \ldots, q_d]$.
\item Set $X = Q_d \Lambda_d^{1/2}$ where $\Lambda_d$ consists of the
  largest $m$ eigenvalues and eigenvectors.
\end{itemize}
We consider the following alternative algorithm:
\begin{itemize}
\item
  Choose $d$ linearly independent columns of $B$ corresponding to
  ``landmark'' points (e.g.~by pivoted QR, though we could also
  consider pivoted Cholesky).  Without loss of generality, we suppose
  the points are permuted so the landmarks appear first.
\item
  Let $B_{11} = R_{11}^T R_{11}$ be the Cholesky factorization of the
  block of $B$ corresponding to interactions between landmark points.
\item
  Let the coordinates be
  \[
    X = \begin{bmatrix} B_{11} \\ B_{21} \end{bmatrix} R_{11}^{-1}
  \]
\end{itemize}
Explain why this algorithm also produces points with the correct
pairwise distances.

\paragraph*{3: Stuck in the middle}
For a fixed $A \in \bbR^{m \times n}$ and full-rank matrices
$L \in \bbR^{m \times k}$ and $R \in \bbR^{k \times n}$,
argue that the choice of $M$ to minimize $\|A-LMR\|_F^2$
is $M = L^\dagger A R^\dagger$.

\end{document}
