\documentclass[12pt, leqno]{article} %% use to set typesize
\usepackage{fancyhdr}
\usepackage[sort&compress]{natbib}
\usepackage[letterpaper=true,colorlinks=true,linkcolor=black]{hyperref}

\usepackage{amsfonts}
\usepackage{amsmath}
\usepackage{amssymb}
\usepackage{color}
\usepackage{tikz}
\usepackage{pgfplots}
\usepackage{listings}
\usepackage{courier}
%\usepackage[utf8]{inputenc}
%\usepackage[russian]{babel}

\lstset{
  numbers=left,
  basicstyle=\ttfamily\footnotesize,
  numberstyle=\tiny\color{gray},
  stepnumber=1,
  numbersep=10pt,
}

\newcommand{\iu}{\ensuremath{\mathrm{i}}}
\newcommand{\bbR}{\mathbb{R}}
\newcommand{\bbC}{\mathbb{C}}
\newcommand{\calV}{\mathcal{V}}
\newcommand{\calW}{\mathcal{W}}
\newcommand{\macheps}{\epsilon_{\mathrm{mach}}}
\newcommand{\matlab}{\textsc{Matlab}}

\newcommand{\ddiag}{\operatorname{diag}}
\newcommand{\fl}{\operatorname{fl}}
\newcommand{\nnz}{\operatorname{nnz}}
\newcommand{\tr}{\operatorname{tr}}
\renewcommand{\vec}{\operatorname{vec}}

\newcommand{\vertiii}[1]{{\left\vert\kern-0.25ex\left\vert\kern-0.25ex\left\vert #1
    \right\vert\kern-0.25ex\right\vert\kern-0.25ex\right\vert}}
\newcommand{\ip}[2]{\langle #1, #2 \rangle}
\newcommand{\ipx}[2]{\left\langle #1, #2 \right\rangle}
\newcommand{\order}[1]{O( #1 )}

\newcommand{\kron}{\otimes}


\newcommand{\hdr}[1]{
  \pagestyle{fancy}
  \lhead{Bindel, Summer 2018}
  \rhead{Numerics for Data Science)}
  \fancyfoot{}
  \begin{center}
    {\large{\bf #1}}
  \end{center}
  \lstset{language=matlab,columns=flexible}  
}


\begin{document}
\hdr{2018-06-19}

\section{Some factorization tools}

We generally seek to approximately factor a data matrix
$A \in \bbR^{m \times n}$ as
\[
  A \approx LMR, \quad
  L \in \bbR^{m \times r},
  M \in \bbR^{r \times r},
  R \in \bbR^{r \times n}.
\]
In this view, we can think of $R$ (or $MR$) as a map from the original
attributes to a smaller set of {\em latent factors}.  Different
factorization methods differ both in the structural constraints on
$L$, $M$, and $R$ and on how the approximation is chosen.  For today,
we will discuss:
\begin{itemize}
\item
  The {\em symmetric eigendecomposition}
  \[
    A = Q \Lambda Q^T
  \]
  where $Q$ is an orthogonal matrix ($Q^T Q = I$) of eigenvectors
  and $\Lambda$ is the diagonal matrix of eigenvalues.  We will
  also consider the decomposition for the {\em generalized} problem,
  in which $Q^T M Q = I$ where $M$ is a symmetric and positive
  definite matrix.
\item
  The {\em singular value decomposition}
  \[
    A = U \Sigma V^T
  \]
  where $U$ and $V$ are orthogonal matrices (in the full version) or
  matrices with orthonormal columns (in the economy version), and
  $\Sigma$ is the diagonal matrix of singular values.  The SVD
  is a veritable Swiss Army knife of matrix factorizations; we will
  use it on its own and as a building block for other methods.
\item
  The {\em pivoted QR factorization}
  \[
    A\Pi = QR
  \]
  where $\Pi$ is a permutation of the columns of $A$, $Q$ is
  orthogonal, and $R$ is upper triangular.  The permutation $\Pi$ is
  chosen to guarantee that the magnitude of the diagonal entries of
  $R$ appear in descending order.  The pivoted QR factorization also
  has a geometric interpretation that is useful in several settings.
\item
  The {\em interpolative decomposition} (ID)
  \[
    A\Pi \approx C \begin{bmatrix} I & T \end{bmatrix}
  \]
  where $C$ is drawn from the columns of $A$ and the entries of $T$
  are bounded in size (no more than two).  The factorization is exact
  when $A$ is low rank.  In many cases, the columns in $C$ are the
  columns that would be chosen by pivoted QR factorization.
\item
  The {\em CUR factorization}
  \[
    A \approx C U R
  \]
  where $C$ and $R$ are drawn from the columns and rows of $A$.
\end{itemize}
The symmetric eigendecomposition and the SVD are both closely tied to
``nice'' continuous optimization problems, and this connection allows
us to make very strong statements about them.  In contrast, pivoted
QR, ID, and CUR involve discrete choices involving column and row
selection, and it is much more difficult to analyze the properties
that arise from these discrete choices.  At the same time, these
latter factorizations are often {\em interpretable} in a way that
the eigendecomposition and SVD are not.

\section{The symmetric eigenvalue problem}

Among their many other uses, in data analysis tasks a real
{\em symmetric} matrix $A$ often represent interactions between unordered
pairs of objects.  We use them to represent edges in undirected
graphs, similarities between objects, covariances of pairs of random
variables, counts of pairs of words that occur together across sets of
documents, and in many other settings.

From the linear algebra perspective, a symmetric matrix also
represents a {\em quadratic form}, i.e.~a purely quadratic function of
many variables.  For a concrete vector space, we write this as
\[
  \phi(x) = \frac{1}{2} x^T A x.
\]
We say $\phi$ (and $A$) is {\em positive definite} if $\phi$ is
positive for any nonzero argument; $\phi$ is
{\em positive semidefinite} if $\phi$ is always non-negative.  Any positive
semidefinite matrix can be represented (not uniquely) as a
{\em Gram matrix} $A = B^T B$; in this case, we have
$\phi(x) = \|Bx\|^2/2$.  If $A$ is a positive semidefinite matrix that
represents similarities between objects, we can think of the rows of
the matrix $B$ as feature vectors for different objects, so that the
similarity between objects is encoded as the dot product between their
feature vectors.  This is a particularly useful perspective if we want
to use matrix factorization methods to cluster similar objects.

A real symmetric matrix is always diagonalizable with
real eigenvalues, and has an orthonormal basis of eigenvectors
$q_1, \ldots, q_n$, so that we can write the eigendecomposition
\[
  A = Q \Lambda Q^T.
\]
Broadly speaking, I tend to distinguish between two related
perspectives on eigenvalues.  The first is the linear map
perspective: $A$ represents an operator mapping a space to itself,
and an eigenvector corresponds to an {\em invariant direction}
for the operator; that is, we read the decomposition as
\[
  A q_i = q_i \lambda_i.
\]  
The second perspective is the quadratic form
perspective: if $A$ is a symmetric matrix representing a quadratic
form, then we read the decomposition as
\begin{align*}
  (Qx)^T A (Qx) &= \sum_{i=1}^n \lambda_i x_i^2 \\
  x^T A x &= \sum_{i=1}^n \lambda_i (q_i^T x)^2.
\end{align*}
We can write the eigenvalues and eigenvectors as the critical points
for the objective function $x^T A x$ subject to $\|x\|^2 = 1$;
in terms of the Lagrangian
\[
  L(x,\lambda) = \frac{1}{2} x^T A x - \frac{\lambda}{2} (x^T x - 1),
\]
we have the KKT conditions
\begin{align*}
  Ax - \lambda x &= 0 \\
 \|x\|^2-1 &= 0
\end{align*}
The eigenvalues and eigenvectors are also the
stationary values and vectors for the {\em Rayleigh quotient}
\[
  \rho_A(x) = \frac{x^T A x}{x^T x}.
\]
If we differentiate $x^T A x - \rho_A x^T x = 0$, we have
\[
  2 \delta x^T (Ax - \rho_A x) - \delta \rho_a (x^T x) = 0
\]
which means that setting $\delta \rho_A = 0$ implies
\[
  Ax-\rho_A(x) x = 0.
\]
The largest eigenvalue is the maximum of the Rayleigh quotient,
and the smallest eigenvalue is the minimum of the Rayleigh quotient.

If $M$ is symmetric and positive definite, it defines an inner product
and an associated Euclidean norm:
\[
  \langle x, y \rangle_M = y^T M x \quad \mbox{ and } \quad
  \|x\|_M^2 = x^T M x = \langle x, x, \rangle_M.
\]
If $A$ is a symmetric matrix and $M$ is symmetric and positive
definite, we might also consider minimizing the quadratic form for
$A$ subject to the constraint $\|x\|_M = 1$.  This gives us the
KKT equations
\begin{align*}
  Ax - Mx \lambda &= 0 \\
  \|x\|_M^2 - 1 &= 0.
\end{align*}
This is an example of a {\em generalized} eigenvalue problem for the
matrix pencil $(A,M)$.  We can write down a full eigendecomposition
for the generalized problem as
\[
  U^T A U = \Lambda \mbox{ where } U^T M U = I.
\]
The eigenvalues of the pencil $(A,M)$ are the same as those
of $M^{-1} A$ or $B^{-T} A B^{-1}$ for any $B$ such that $M = B^T B$.
We have already described one example where this type of generalized
decomposition is useful when we discussed ``fisherfaces'' in the last
lecture.

The optimization problem associated with the symmetric eigenvalue
problem is {\em not} convex -- far from it!  But it is a nice
optimization problem nonetheless.  It has saddle points, but the only
local minima and maxima are also global minima and maxima, so even
standard optimizers are not prone to getting stuck in a local minimum.
But there are also specialized iterations for solving the symmetric
eigenvalue problem that are very efficient.  While the details of
these methods are a topic for a different class, it is worth sketching
just a few ideas in order to understand the different types of solvers
available and their complexities:
\begin{itemize}
\item
  Most of the {\em direct} solvers\footnote{%
    This is a bit of a fib in that all eigenvalue solvers are
    iterative.  We say these methods are ``direct'' because we
    only need a constant number of iterations per eigenvalue
    to find the eigenvalues to nearly machine precision.}
  for the symmetric eigenvalue problem go through two stages.  First,
  we compute
  \[
    A = U T U^T
  \]
  where $U$ is an orthogonal matrix and $T$ is tridiagonal.  Then we
  compute the eigenvalues and eigenvectors of the resulting
  tridiagonal matrix.  The asymptotically fastest of the tridiagonal
  eigenvalue solvers (the ``grail'' code) takes $O(n)$ time to find
  all eigenvalues and $O(n^2)$ to find all eigenvectors;
  but the cost of reducing $A$ to a tridiagonal form is
  $O(n^3)$.  When we call {\tt eig} in MATLAB, or related solvers in
  other languages, this is the algorithm we use.
\item
  There are also several {\em iterative} methods for computing a few
  eigenvalues and eigenvectors of a symmetric matrix.  The simplest
  methods are the {\em power iteration} and {\em subspace iteration},
  which you may have seen in a previous class; unfortunately, these
  methods do not converge very quickly.  The {\em Lanczos} method is
  the main workhorse of sparse eigenvalue solvers, and the method you
  will use if you call {\tt eigs} in MATLAB, or related subroutines in
  other languages.  Like power iteration and subspace iteration, the
  Lanczos method requires only matrix-vector multiplications, and so
  it can be very efficient when we want to compute matrix-vector
  products.
\end{itemize}
The Lanczos method is good at computing a few of the largest and
smallest eigenvalues of a symmetric matrix (the {\em extremal
  eigenvalues}), together with the corresponding eigenvectors.  It is
not as good at computing {\em interior} eigenvalues without some help,
usually in the form of a {\em spectral transformation}.  But for many
applications in data science, we are content to look at the extremal
eigenvalues and vectors, so we will not discuss the more difficult
interior case any further.

\section{The singular value decomposition}

The singular value decomposition of $A \in \bbR^{m \times n}$ is
\[
  A = U \Sigma V^T
\]
where $U$ and $V$ have orthonormal columns and $\Sigma$ has the
non-negative {\em singular values} of $A$ in descending order on the
diagonal (and zeros elsewhere).  We sometimes distinguish between the
``full'' SVD in which $U$ and $V$ are square and $\Sigma$ is the same
size as $A$; and the ``economy'' SVD, in which $\Sigma$ is square and
one of $U$ or $V$ is the same size as $A$.

Like the symmetric eigenvalue decomposition, the singular value
decomposition can be viewed in terms of an associated optimization
problem:
\[
  \mbox{minimize } \|Av\|_2 \mbox{~s.t.~} \|v\|_2 = 1.
\]
We change nothing by squaring the norms to get
\[
  \mbox{minimize } \|Av\|_2^2 \mbox{~s.t.~} \|v\|_2^2 = 1,
\]
but we can rewrite this as
\[
  \mbox{minimize } v^T A^T A v \mbox{~s.t.~} v^T v = 1,
\]
which we recognize as exactly the optimization formulation of the
maximum eigenvalue (and eigenvector) of $A^T A$.  More generally,
we have
\[
  A^T A = V \Sigma^2 V^T.
\]
where the eigenvectors of $A^T A$ are the constrained critical points
for the optimization problem and the (non-negative) eigenvalues in
$\Sigma^2$ are the corresponding eigenvectors.  Having computed the
vectors $V$, we write
\[
  A V = U \Sigma
\]
where we know the columns of $U$ each have unit Euclidean norm.  To
see that the columns of $U$ are actually orthonormal, we write
\[
  AA^T = AV V^T A^T = U \Sigma^2 U^T.
\]
And this is exactly the eigenvalue decomposition of $AA^T$!

We say a function $f : \bbR^{m \times n} \rightarrow \bbR$ is
{\em orthogonally invariant} (or {\em unitarily invariant}\footnote{%
  Unitarily invariant covers the complex case}
if $f(Q_1 A Q_2) = f(A)$ for any orthogonal matrices $Q_1$ and $Q_2$.
Any unitarily invariant function can be written in terms of the
singular values of $A$.  Among the most important unitarily invariant
functions are the {\em Ky-Fan} norms, which are just the $\ell^p$
norms of the vector of singular values.  We generally only care about three
of these norms:
\begin{itemize}
\item The {\em operator 2-norm} (or {\em spectral norm}) is
  $\|A\|_2 = \sigma_1$; this is the Ky-Fan norm for $p = \infty$.
\item The {\em Frobenius norm} satisfies
  $\|A\|_F^2 = \sum_i \sigma_i^2$; that is, $\|A\|_F$ is the
  Ky-Fan norm for $p = 2$.
\item The {\em nuclear norm} satisfies $\|A\|_* = \sum_i \sigma_i$;
  that is, $\|A\|_*$ is the Ky-Fan norm for $p = 1$.  We will see
  more of this norm when we talk about matrix completion on Friday.
\end{itemize}
The Eckart-Young theorem tells us that the {\em best} rank $k$ approximation
to $A$ in the spectral norm or the Frobenius norm is the truncated
singular value decomposition.  In fact, the theorem holds for any of
the Ky-Fan norms (so it is true in the nuclear norm as well).

The singular value decomposition goes by many names and has many close
relatives.  It is sometimes called the
{\em proper orthogonal decomposition} of a data matrix;
in statistics, it is the basis for {\em principal component analysis};
and in the study of stochastic processes,
it is the {\em Karhuenen-Lo\`eve} decomposition.
But in some contexts, the SVD is not applied directly to our data
matrix $A$, but is instead applied to a transformed matrix.  For
example, if the columns of $A$ represent samples of different random
variables (and the rows represent experiments), we would typically
look at the SVD of the {\em centered} matrix $A-e\mu^T$ where $\mu$ is
the vector of column means and $e$ is the vector of all ones.  And in
other settings, we might look at the matrix of $z$-scores, which are
obtained by normalizing the Euclidean lengths of the vectors of the
centered matrix.

\section{Pivoted QR and pivoted Cholesky}

The SVD provides optimal low-rank approximations to data matrices, but
those approximations are not particularly easy to interpret.  We
therefore turn now to low-rank factorizations in which the factors are
formed from subsets of the columns or rows of the data matrix.

We described the {\em pivoted QR decomposition} briefly in our discussion
of least squares.  The idea in pivoted QR is to permute the columns of
the data matrix so that the diagonal entries of the QR factorization
of the pivoted matrix appear in descending magnitude, i.e.
\[
  A \Pi = Q R, \quad r_{ii} \geq r_{i+1,i+1} \mbox{ for all } i.
\]
The column order is computed in a greedy fashion as follows.  At the
first step of the iteration, we choose the column of $A$ with the
largest Euclidean norm; we then scale it by the norm $r_{11}$ in order
to get the first column of $Q$:
\[
  A \Pi_{:,1} = q_1 r_{11}.
\]
Next, we find the column of $A$ with the largest component orthogonal
to $q_1$, and write it as
\[
  A \Pi_{:,2} = q_1 r_{12} + q_2 r_{22}
\]
where $r_{12}$ is the extent to which the vector projects onto $q_1$,
$q_2$ is the direction of the residual component orthogonal to $q_1$,
and $r_{22}$ is the magnitude of that second component.  And we
proceed in a similar fashion until we run out of columns or until
all the remaining residuals are tiny.

Though we have described this decomposition in terms of explicit
orthogonalizations at each step (as in the Gram-Schmidt procedure), in
practice we would usually use an alternate algorithm based on
orthogonal transformations.  However, the algorithm we have described
is useful for reasoning about a special property of the algorithm: the
columns it selects are all on the {\em convex hull} of the set of
columns of $A$.  In general, for any vector $u$, the column for which
$|u^T A|$ is largest will lie on the convex hull of the columns of
$A$; and by design, $q_i^T A \Pi = R_{i,:}$ has its largest entry on
the diagonal.  This fact is key to the use of pivoted QR in some
algorithms for {\em separable} non-negative matrix factorization, as
we will discuss next time.

Closely related to the pivoted QR algorithm is the {\em pivoted
  Cholesky} algorithm: for a positive semi-definite matrix $A$,
pivoted Cholesky computes
\[
  \Pi^T A \Pi = R^T R
\]
where the diagonal entries of $R$ have descending magnitude.  Like
pivoted QR, pivoted Cholesky is a greedy algorithm, and at each step
it chooses the next pivot row/column based on the magnitude of a
``residual'' diagonal\footnote{This is really the diagonal of the
  Schur complement, for those of you who may have seen Schur
  complements in a discussion of Gaussian elimination in an earlier
  class.}.
In exact arithmetic, pivoted Cholesky on a Gram matrix $B^T B$
computes the same permutation and $R$ factor as pivoted QR on $B$.

\section{Interpolative decomposition and CUR}

An {\em interpolative decomposition} is a decomposition of the form
\[
  A \Pi \approx C \begin{bmatrix} I & T \end{bmatrix}
\]
where $C$ is a subset of the columns of $A$.  One way to get the
interpolative decomposition is via truncated pivoted QR; is we write
\[
  A \Pi \approx
  Q \begin{bmatrix} R_1 & R_2 \end{bmatrix}
\]
where $R_1 \in \bbR^{k \times k}$ is the leading submatrix in the
rectangular $R$ factor, then
\[
  A \Pi =
  Q R_1 \begin{bmatrix} I & R_1^{-1} R_2 \end{bmatrix} =
  C \begin{bmatrix} I & T \end{bmatrix}
\]
Unfortunately, if we only use truncated pivoted QR, we cannot
guarantee that we are very close to the best rank $k$ approximation;
nor can we guarantee that the entries of $T$ are nice.  It is possible
to show that there is {\em some} permutation such that the entries
of $T$ are at most 2 in magnitude,
the singular values of $\begin{bmatrix} I & T \end{bmatrix}$
lie between $1$ and $1+\sqrt{k(n-k)}$, and the approximation
error is within a factor of $1+\sqrt{k(n-k)}$ of the best possible.
But pivoted QR might not find the best choice.  Fortunately, we can
get close to optimal by either a post-processing algorithm that
iteratively swaps new columns for the original selection of columns in
order to reduce the size of elements in $T$; or we can use randomized
algorithms.

In the interpolative decomposition, we choose a subset of the columns
of $A$ as the basis for our approximation.  In the CUR decomposition,
we choose both columns {\em and rows}, i.e.
\[
  A \approx C U R
\]
where $C$ and $R$ are drawn from the rows and columns of $A$.
Given a choice of $C$ and $R$, we find the optimal choice of $U$
by least squares:
\[
  U = C^\dagger A R^\dagger.
\]
Again, though, we have a problem: how should we select the columns and
rows to use?  A simple approach is to run pivoted QR on both the rows
and columns, potentially with ``swapping'' algorithms of the type used
in ID.  An alternative approach is to choose rows and columns based on
a randomized scheme using approximate ``leverage scores'' to determine
the importance of choosing a given column or row for the final
factorization.

\end{document}
