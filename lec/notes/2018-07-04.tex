\documentclass[12pt, leqno]{article} %% use to set typesize
\usepackage{fancyhdr}
\usepackage[sort&compress]{natbib}
\usepackage[letterpaper=true,colorlinks=true,linkcolor=black]{hyperref}

\usepackage{amsfonts}
\usepackage{amsmath}
\usepackage{amssymb}
\usepackage{color}
\usepackage{tikz}
\usepackage{pgfplots}
\usepackage{listings}
\usepackage{courier}
%\usepackage[utf8]{inputenc}
%\usepackage[russian]{babel}

\lstset{
  numbers=left,
  basicstyle=\ttfamily\footnotesize,
  numberstyle=\tiny\color{gray},
  stepnumber=1,
  numbersep=10pt,
}

\newcommand{\iu}{\ensuremath{\mathrm{i}}}
\newcommand{\bbR}{\mathbb{R}}
\newcommand{\bbC}{\mathbb{C}}
\newcommand{\calV}{\mathcal{V}}
\newcommand{\calW}{\mathcal{W}}
\newcommand{\macheps}{\epsilon_{\mathrm{mach}}}
\newcommand{\matlab}{\textsc{Matlab}}

\newcommand{\ddiag}{\operatorname{diag}}
\newcommand{\fl}{\operatorname{fl}}
\newcommand{\nnz}{\operatorname{nnz}}
\newcommand{\tr}{\operatorname{tr}}
\renewcommand{\vec}{\operatorname{vec}}

\newcommand{\vertiii}[1]{{\left\vert\kern-0.25ex\left\vert\kern-0.25ex\left\vert #1
    \right\vert\kern-0.25ex\right\vert\kern-0.25ex\right\vert}}
\newcommand{\ip}[2]{\langle #1, #2 \rangle}
\newcommand{\ipx}[2]{\left\langle #1, #2 \right\rangle}
\newcommand{\order}[1]{O( #1 )}

\newcommand{\kron}{\otimes}


\newcommand{\hdr}[1]{
  \pagestyle{fancy}
  \lhead{Bindel, Summer 2018}
  \rhead{Numerics for Data Science)}
  \fancyfoot{}
  \begin{center}
    {\large{\bf #1}}
  \end{center}
  \lstset{language=matlab,columns=flexible}  
}


\begin{document}
\hdr{2018-07-04}

% Semi-supervised learning task (two class)
% Discrete energy optimization
% Continuous relaxation and the Laplacian
% Laplacian as a precision matrix
% Eigenfunction featurization
% Electrical analogies
% Kernels and distances
% Extending to multi-class
% Alternatives: the heat kernel
% Laplacian solves as a building block
% Two types of graphs and two types of solvers

\section{Semi-supervised learning}

Suppose we have a collection of objects that we want to classify
one of two ways.  Given some labeled examples, how should we label
the remaining example?  This is a standard
{\em semi-supervised learning} task.

In order to make sense of this, we need some notion of the similarity
between pairs of objects, with the understanding that similar objects
should probably have the same label.  We encode this intuition in a
weighted graph, with weighted adjacency matrix $A$ encoding the
similarity and $L$ the weighted Laplacian.  Suppose $x \in \{0,1\}^n$,
with the value of each element of $x$ indicating the class of the
corresponding object.  Then we seek
\[
  \mbox{minimize } x^T L x \mbox{ s.t.~} x_1 = y
\]
where $x_1$ is the subvector of labels for which we have training
data, which we order to the start of $x$ without loss of generality.

\end{document}
